 Network zoning is an important way of securing assets in a network. The process of creating and managing network zones, as well as transitions between these zones has historically been proven to be hard. Especially in data centers of large cloud providers, network zoning is extremely important, since it is a multi-tenancy environment. The new inter-domain network zoning architecture named MONDRIAN simplifies the management of network zones and transitions between them. However, it is designed to be used in corporate networks, meaning that it’s not applicable to data center networks. 

 %Methods
 %  - Design, implement and evaluate a PoC
 
 In this thesis we redesigned MONDRIAN to be usable and beneficial in the context of data center network security. We created a \acl{PoC} implementation, which leverages the logically centralized property of MONDRIAN while being optimized for large scale data centers. We rigorously evaluated the functionality, security and performance of the newly designed MONDRIAN variant.

 %Results
 %  - Functionality achieved, Security as on the same level as original MONDRIAN and performance ok for PoC implementation. Scalability and integrates well in DCs

 The results show that with the new design, we achieve the full functionality of MONDRIAN and are therefore able to provide equally strong security and manageability benefits. The performance is on a decent level for a \acl{PoC} implementation and scalability is shown in several experiments.
 
 %Conclusion
 %  - MONDRIAN is feasible in the context of DCs thanks to the new design
 %  - Clear benefits exist
 The thesis shows the feasibility of employing a logically centralized network zoning architecture like MONDRIAN in the context of data center networking. It proposes a completely redesigned architecture, which is optimized for the use in modern data centers and provides data center operators with the benefits MONDRIAN has to offer.