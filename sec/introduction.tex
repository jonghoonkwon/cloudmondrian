% \chapter{Introduction}
%Motivation
Network zoning in data centers of course already exists today. However, current approaches require us to configure multiple components such as routers, firewalls and \acsp{VPN} (\aclp{VPN}). Some of these configurations are site specific, while others need to be consistent across sites. This combined with the fact that there is typically no central database containing all the zone transition polices makes the management of network zones in data centers extremely difficult. %Data center operators usually need to decide if they want to sacrifice either \textbf{functionality} by not allowing zone transitions at all, \textbf{flexibility} by not moving hosts around and not changing once established zone transition policies or \textbf{security} by losing the oversight over the complexity of the configuration and therefore risking misconfigurations.
Data center operators usually need to decide if they want to sacrifice either \textbf{functionality}, \textbf{flexibility} or \textbf{security}. Sacrificing functionality would mean to not allow zone transitions at all. Flexibility could be sacrificed by not moving hosts around in data centers and not changing once established zone transition policies. Compromising security would happen once the data center operators inevitably lose oversight over the complexity of the configuration and therefore misconfigurations would be unavoidable.

By using MONDRIAN, an inter-domain network zoning architecture, in a data center, the data center operator can save expenses due to the egress-filtering MONDRIAN provides. No unauthorized traffic will traverse the expensive inter-domain links. The security will not only be increased due to the simplified manageability but it will also be easily understandable and verifiable by customers of the data center. The operator can simply present the customers a set of policies, which is relevant to them. Without MONDRIAN, the data center operator would need to disclose detailed configuration information about the networking setup. This would be too hard to understand for the customers and would most likely also just be valid for a short period of time, since data center networks change constantly. MONDRIAN configurations and zone transition policies stay exactly the same even if hosts are added, removed or moved around both within and between sites. This makes data centers much more scalable thanks to MONDRIAN.

Considering the aforementioned benefits a MONDRIAN deployment in a data center provides, should make it clear why there's a huge potential for applying MONDRIAN in data centers.

% Main idea / contribution

This master thesis project is about adapting the design of MONDRIAN, such that it is usable in modern, large-scale data centers. Data center networking is fundamentally different from traditional networking like it is performed in corporate networks, which is what MONDRIAN has originally been designed for. Data center networks have completely different topologies than corporate networks. They employ completely different technologies, which are designed to be integratable into the heavily virtualized environment we encounter in data centers. Furthermore, performance requirements are much higher than in corporate networks since resources are much more monetized. 

While considering these factors, we redesign MONDRIAN to cope with the challenges we face in a data center. We turn the functionality of a \acl{TP} into a \acl{VNF}, by partially using \acs{SDN} (\acl{SDN}) and partially using \acs{VM} (\acl{VM}) / container-based \acl{NFV}.

% results - take away message
The resulting design is implemented as a \acl{PoC} implementation and is then carefully evaluated to guarantee functionality, security as well as decent performance. The findings of the evaluations were that the new MONDRIAN design fulfills the requirements and could therefore be implemented as a production-level software package, which would be usable and highly beneficial for data center operators. It would allow them to take advantage of the properties a MONDRIAN deployment provides while only interfering minimally with their existing architectures, their topology and their technology portfolio. 

% contribution
The major contribution of this master thesis project is the redesign of MONDRIAN for the use in modern data centers and the \acl{PoC} implementation, consisting of all software components used by this MONDRIAN version.
%\todo{rewrite without sections (motivation / main idea / contribution / results) ... do that in the end}